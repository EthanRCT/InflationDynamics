\documentclass{article}
\usepackage{amsmath, amssymb, amsfonts, amsthm, mathtools, bm}
\usepackage{geometry}
\usepackage{hyperref}
\usepackage[english]{babel} % or \usepackage{polyglossia}
\usepackage{graphicx}

\title{Blending SIR and Predator-Prey Models to Predict the Labor Market}

\author{Jason Vasquez \and Dylan Skinner \and Benjamin Mcmullin \and Ethan Crawford}

\date{December 5 2023}

\begin{document}

\maketitle

We give permission for this work to be shared by ACME Volume 4 instructors and anybody else who may have reasonable motivation

\begin{center}
    \textbf{Abstract}
\end{center}

The labor market, including the unemployment rate and the amount of workers looking for jobs, can have a large impact on the economny.
The more people employed means more money being spent, which in turn means more money being made. 
Furthermore, rise in unemployment can lead to a recession. Being able to predict the labor market can help us prepare for a recession and help us understand the economy better.
In this paper, we adapt an SIR model to model the amount of employed, unemployed, and retired individuals.
Furthermore, we use a quasi predator-prey model to illustrate the oscillation of the two industries commonly known as white-collar and blue-collar.

\section{Background/Motivation}



% Rest of your document goes here

\end{document}
